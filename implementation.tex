\chapter{Implementation}
\label{cha:implementation}
\vspace{0.4 cm} 

Write about the implementation of the system \dots

This chapter presents how the components of the system have been implemented.
Starting by describing the implementation of the data collector on a Raspberry Pi and then explaining how the Mosquitto MQTT Broker, the MongoDB database and the two Back-End parts, one for receiving and storing the data and the other one for analyzing the data, has been implemented on Docker containers. After this chapter, it will be clear how the system was implemented for the testing and validation, that are discussed in the next chapter.


\section{Data collector implementation}
\label{sec:collector}
\vspace{0.2 cm} 

Write how the sniffer has been implemented on the Raspberry Pi \dots

Data collector on a Raspberry Pi model 2B using:
\begin{itemize}
  \item EDUP 802.11n Wi-Fi Dongle (Range $\sim$ 10 meters) + Scapy for Wi-Fi packets sniffing
  \item RTC Board for timestamp detection
  \item BLAKE2s for MAC address anonymization
  \item Netifaces for managing the connection
  \item Paho-MQTT for the MQTT transmission
\end{itemize}


% \section{Implementation of broker MQTT}
% \label{sec:broker}
% \vspace{0.2 cm} 

% Write how the broker MQTT has been implemented \dots


\section{Server implementation}
\label{sec:server}
\vspace{0.2 cm} 

Write how the Back-End for analysis has been implemented \dots

Docker Containers:
\begin{itemize}
  \item MQTT broker: Mosquitto\footnote{ website: \url{https://hub.docker.com/_/eclipse-mosquitto} }
  \item Database: MongoDB\footnote{ website: \url{https://hub.docker.com/_/mongo} }
  \item Receiver with MQTT client using paho-MQTT for receiving the data and Mongo client using pymongo for storing data on MongoDB
  \item Analyzer with Mongo client using pymongo and pandas to read and manage the data
  \item Sklearn is used to analyze this data and to implement the machine learning part
\end{itemize}
